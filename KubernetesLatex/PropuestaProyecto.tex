\documentclass[letterpaper,12pt]{article}
\usepackage{tabularx} % extra features for tabular environment
\usepackage{amsmath}  % improve math presentation
\usepackage{graphicx} % takes care of graphic including machinery
\usepackage[margin=1in,letterpaper]{geometry} % decreases margins
\usepackage{cite} % takes care of citations
\usepackage[final]{hyperref} % adds hyper links inside the generated pdf file
\hypersetup{
	colorlinks=true,       % false: boxed links; true: colored links
	linkcolor=blue,        % color of internal links
	citecolor=blue,        % color of links to bibliography
	filecolor=magenta,     % color of file links
	urlcolor=blue         
}
\usepackage{blindtext}
%++++++++++++++++++++++++++++++++++++++++


\begin{document}

\title{Kubernetes y sistemas distribuidos}
\author{Cynthia calixtro, Daniel Hidalgo, Nelson Sanabio}
\date{}
\maketitle
% Articulos científicos relevantes
% Propuesta de proyecto a realizar / utilizar
\section{Introduction}

\section{Background}


\section{Conclusions}
Here you briefly summarize your findings. Did you learn any new physics? Was everything as expected?

\section{Future Work}
Since you had limited time to work on this project, what questions are left outstanding? What would be your next steps? 


\begin{thebibliography}{99}

\bibitem{melissinos}
A.~C. Melissinos and J. Napolitano, \textit{Experiments in Modern Physics},
(Academic Press, New York, 2003).

\bibitem{Cyr}
N.\ Cyr, M.\ T$\hat{e}$tu, and M.\ Breton,
% "All-optical microwave frequency standard: a proposal,"
IEEE Trans.\ Instrum.\ Meas.\ \textbf{42}, 640 (1993).

\bibitem{Wiki} \emph{Expected value},  available at
\texttt{http://en.wikipedia.org/wiki/Expected\_value}.

\end{thebibliography}


\end{document}